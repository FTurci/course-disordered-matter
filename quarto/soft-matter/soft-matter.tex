% Options for packages loaded elsewhere
\PassOptionsToPackage{unicode}{hyperref}
\PassOptionsToPackage{hyphens}{url}
\PassOptionsToPackage{dvipsnames,svgnames,x11names}{xcolor}
%
\documentclass[
  letterpaper,
  DIV=11,
  numbers=noendperiod]{scrartcl}

\usepackage{amsmath,amssymb}
\usepackage{iftex}
\ifPDFTeX
  \usepackage[T1]{fontenc}
  \usepackage[utf8]{inputenc}
  \usepackage{textcomp} % provide euro and other symbols
\else % if luatex or xetex
  \usepackage{unicode-math}
  \defaultfontfeatures{Scale=MatchLowercase}
  \defaultfontfeatures[\rmfamily]{Ligatures=TeX,Scale=1}
\fi
\usepackage{lmodern}
\ifPDFTeX\else  
    % xetex/luatex font selection
\fi
% Use upquote if available, for straight quotes in verbatim environments
\IfFileExists{upquote.sty}{\usepackage{upquote}}{}
\IfFileExists{microtype.sty}{% use microtype if available
  \usepackage[]{microtype}
  \UseMicrotypeSet[protrusion]{basicmath} % disable protrusion for tt fonts
}{}
\makeatletter
\@ifundefined{KOMAClassName}{% if non-KOMA class
  \IfFileExists{parskip.sty}{%
    \usepackage{parskip}
  }{% else
    \setlength{\parindent}{0pt}
    \setlength{\parskip}{6pt plus 2pt minus 1pt}}
}{% if KOMA class
  \KOMAoptions{parskip=half}}
\makeatother
\usepackage{xcolor}
\setlength{\emergencystretch}{3em} % prevent overfull lines
\setcounter{secnumdepth}{-\maxdimen} % remove section numbering
% Make \paragraph and \subparagraph free-standing
\makeatletter
\ifx\paragraph\undefined\else
  \let\oldparagraph\paragraph
  \renewcommand{\paragraph}{
    \@ifstar
      \xxxParagraphStar
      \xxxParagraphNoStar
  }
  \newcommand{\xxxParagraphStar}[1]{\oldparagraph*{#1}\mbox{}}
  \newcommand{\xxxParagraphNoStar}[1]{\oldparagraph{#1}\mbox{}}
\fi
\ifx\subparagraph\undefined\else
  \let\oldsubparagraph\subparagraph
  \renewcommand{\subparagraph}{
    \@ifstar
      \xxxSubParagraphStar
      \xxxSubParagraphNoStar
  }
  \newcommand{\xxxSubParagraphStar}[1]{\oldsubparagraph*{#1}\mbox{}}
  \newcommand{\xxxSubParagraphNoStar}[1]{\oldsubparagraph{#1}\mbox{}}
\fi
\makeatother


\providecommand{\tightlist}{%
  \setlength{\itemsep}{0pt}\setlength{\parskip}{0pt}}\usepackage{longtable,booktabs,array}
\usepackage{calc} % for calculating minipage widths
% Correct order of tables after \paragraph or \subparagraph
\usepackage{etoolbox}
\makeatletter
\patchcmd\longtable{\par}{\if@noskipsec\mbox{}\fi\par}{}{}
\makeatother
% Allow footnotes in longtable head/foot
\IfFileExists{footnotehyper.sty}{\usepackage{footnotehyper}}{\usepackage{footnote}}
\makesavenoteenv{longtable}
\usepackage{graphicx}
\makeatletter
\newsavebox\pandoc@box
\newcommand*\pandocbounded[1]{% scales image to fit in text height/width
  \sbox\pandoc@box{#1}%
  \Gscale@div\@tempa{\textheight}{\dimexpr\ht\pandoc@box+\dp\pandoc@box\relax}%
  \Gscale@div\@tempb{\linewidth}{\wd\pandoc@box}%
  \ifdim\@tempb\p@<\@tempa\p@\let\@tempa\@tempb\fi% select the smaller of both
  \ifdim\@tempa\p@<\p@\scalebox{\@tempa}{\usebox\pandoc@box}%
  \else\usebox{\pandoc@box}%
  \fi%
}
% Set default figure placement to htbp
\def\fps@figure{htbp}
\makeatother

\KOMAoption{captions}{tableheading}
\makeatletter
\@ifpackageloaded{caption}{}{\usepackage{caption}}
\AtBeginDocument{%
\ifdefined\contentsname
  \renewcommand*\contentsname{Table of contents}
\else
  \newcommand\contentsname{Table of contents}
\fi
\ifdefined\listfigurename
  \renewcommand*\listfigurename{List of Figures}
\else
  \newcommand\listfigurename{List of Figures}
\fi
\ifdefined\listtablename
  \renewcommand*\listtablename{List of Tables}
\else
  \newcommand\listtablename{List of Tables}
\fi
\ifdefined\figurename
  \renewcommand*\figurename{Figure}
\else
  \newcommand\figurename{Figure}
\fi
\ifdefined\tablename
  \renewcommand*\tablename{Table}
\else
  \newcommand\tablename{Table}
\fi
}
\@ifpackageloaded{float}{}{\usepackage{float}}
\floatstyle{ruled}
\@ifundefined{c@chapter}{\newfloat{codelisting}{h}{lop}}{\newfloat{codelisting}{h}{lop}[chapter]}
\floatname{codelisting}{Listing}
\newcommand*\listoflistings{\listof{codelisting}{List of Listings}}
\makeatother
\makeatletter
\makeatother
\makeatletter
\@ifpackageloaded{caption}{}{\usepackage{caption}}
\@ifpackageloaded{subcaption}{}{\usepackage{subcaption}}
\makeatother

\usepackage{bookmark}

\IfFileExists{xurl.sty}{\usepackage{xurl}}{} % add URL line breaks if available
\urlstyle{same} % disable monospaced font for URLs
\hypersetup{
  pdftitle={Soft Condensed Matter Physics},
  pdfauthor={Nigel B. Wilding, Francesco Turci},
  colorlinks=true,
  linkcolor={blue},
  filecolor={Maroon},
  citecolor={Blue},
  urlcolor={Blue},
  pdfcreator={LaTeX via pandoc}}


\title{Soft Condensed Matter Physics}
\author{Nigel B. Wilding, Francesco Turci}
\date{Invalid Date}

\begin{document}
\maketitle

\renewcommand*\contentsname{Table of contents}
{
\hypersetup{linkcolor=}
\setcounter{tocdepth}{3}
\tableofcontents
}

\pandocbounded{\includegraphics[keepaspectratio]{soft-matter_files/mediabag/2025_01_27_a9af280ad.jpg}}
(1) Introduction and overview
\pandocbounded{\includegraphics[keepaspectratio]{soft-matter_files/mediabag/2025_01_27_a9af280ad1.jpg}}

Part 1: COLLOIDS (2) Brownian motion (3) Interacting colloidal
suspensions
\pandocbounded{\includegraphics[keepaspectratio]{soft-matter_files/mediabag/2025_01_27_a9af280ad12.jpg}}

Part 2: POLYMERS (4) Chemical structure (5) Polymer conformations (6)
Concentrated polymer solutions
\pandocbounded{\includegraphics[keepaspectratio]{soft-matter_files/mediabag/2025_01_27_a9af280ad123.jpg}}

Part 3: SURFACTANTS

\begin{enumerate}
\def\labelenumi{(\arabic{enumi})}
\setcounter{enumi}{6}
\tightlist
\item
  Chemical structure
\item
  Self-assembled structures
  \pandocbounded{\includegraphics[keepaspectratio]{soft-matter_files/mediabag/2025_01_27_a9af280ad1234.jpg}}
\end{enumerate}

Part 4: GLASSES (9) Glass forming systems (10) Relaxation time and
viscosity

\subsection{INTRODUCTION}\label{introduction}

Soft condensed matter physics, or macromolecular physics, or the physics
of complex fluids, is the study of colloids, polymers and surfactants.
All these systems contain structural units in the colloidal length
scale,
\(1 \mathrm{~nm} \lesssim \mathrm{R} \lesssim 1 \mu \mathrm{~m}\).

\subsubsection{Systems and definitions}\label{systems-and-definitions}

\begin{itemize}
\tightlist
\item
  A colloid consists of a disperse phase (solid, liquid or gas
  `particles') distributed in a finely divided state in a dispersion
  medium (solid, liquid or gas `solvent').
\end{itemize}

\begin{longtable}[]{@{}
  >{\raggedright\arraybackslash}p{(\linewidth - 6\tabcolsep) * \real{0.2500}}
  >{\raggedright\arraybackslash}p{(\linewidth - 6\tabcolsep) * \real{0.2500}}
  >{\raggedright\arraybackslash}p{(\linewidth - 6\tabcolsep) * \real{0.2500}}
  >{\raggedright\arraybackslash}p{(\linewidth - 6\tabcolsep) * \real{0.2500}}@{}}
\toprule\noalign{}
\begin{minipage}[b]{\linewidth}\raggedright
disp. medium disp. phase
\end{minipage} & \begin{minipage}[b]{\linewidth}\raggedright
solid
\end{minipage} & \begin{minipage}[b]{\linewidth}\raggedright
liquid
\end{minipage} & \begin{minipage}[b]{\linewidth}\raggedright
gas
\end{minipage} \\
\midrule\noalign{}
\endhead
\bottomrule\noalign{}
\endlastfoot
solid & solid suspension: pigmented plastics, stained glass, ruby glass,
opal, pearl & sol, coll.susp.: metal sol, toothpaste, paint, ink, clay
slurries, mud & solid aerosol: smoke, dust \\
liquid & solid emulsion: bituminous road paving, ice cream & emulsion:
milk, mayonnaise, butter, pharma- ceutical creams, digestion & log,
mist, tobacco \\
smoke, hair spray & & & \\
fas & solid foam: zeolites, expanded polystyrene, `silica gel' & foam:
froths, soap foam, fire-extinguisher & \\
gas & & & \\
\end{longtable}

\begin{itemize}
\tightlist
\item
  A polymer is a large molecule made up of many (hundreds to millions)
  repeats of a single subunit (`monomer'). It is kept together by
  covalent chemical bonds. Different architectures exist: linear-chain
  homopolymers, (block) copolymers, starpolymers, dendrimers etc. We
  concentrate on linear chains, which also represents the main building
  block of the other structures. Examples: polyethylene, cellulose, DNA,
  proteins.
  \pandocbounded{\includegraphics[keepaspectratio]{soft-matter_files/mediabag/2025_01_27_a9af280ad12345.jpg}}
  \pandocbounded{\includegraphics[keepaspectratio]{soft-matter_files/mediabag/2025_01_27_a9af280ad123456.jpg}}
\end{itemize}

- A surfactant (SURFace ACTive AgeNT) is a special kind of molecule. In
the simplest case, it has a water-loving (hydrophilic) `head group', and
a water-hating (hydrophobic) `tail'. Above a certain concentration they
spontaneously selfassemble into a wide variety of structures, the
simplest being a spherical micelle. They associate by (rather weak)
physical interactions (`association colloids'). Examples:
\pandocbounded{\includegraphics[keepaspectratio]{soft-matter_files/mediabag/2025_01_27_a9af280ad1234567.jpg}}
Sodium Dodecylsulfate (SDS), lipids.

Note: In general, when in bulk form, the substance forming colloidal
particles is quite insoluble in the dispersion medium. However, when the
bulk material is broken down into very small pieces it can be dispersed.
Such a system is called a lyophobic (`solvent-avoiding') colloid. Some
colloidal particles themselves consist of polymers or surfactant
assemblies which can easily be dispersed and are thus labelled lyophilic
(`solvent-loving'). Mostly when we talk about ``colloids'', we do not
care about the `internal' structure of the particles, while we are in
general interested in the structure of polymers or surfactant
assemblies.

\begin{itemize}
\tightlist
\item
  Many liquid-state soft matter system, particular colloids and polymers
  can be cooled to very low temperatures without crystallizing. At these
  low temperatures, the typical relaxation times of the particles
  becomes very high, even on experimental time scales. Thus the
  liquid-like disorder is ``frozen-in'' and we refer to the system as a
  glass.
\end{itemize}

\subsubsection{Macromolecules: principal distinguishing
features}\label{macromolecules-principal-distinguishing-features}

\paragraph{\texorpdfstring{Size Range (
\(1 \mathrm{~nm} \lesssim R \lesssim 1 \mu \mathrm{~m}\)
)}{Size Range ( 1 \textbackslash mathrm\{\textasciitilde nm\} \textbackslash lesssim R \textbackslash lesssim 1 \textbackslash mu \textbackslash mathrm\{\textasciitilde m\} )}}\label{size-range-1-mathrmnm-lesssim-r-lesssim-1-mu-mathrmm}

Lower limit (about 1 nm ): This ensures that the size of the dispersed
particles is much larger than the size of the molecules forming the
dispersion medium. This not only distinguishes colloids from ordinary
(true) solutions, but also results in a `welldefined' interface between
the particles and the surrounding medium. Furthermore, (most of the
time) this allows us both, to treat the dispersion medium as a continuum
and to avoid quantum mechanics.

\paragraph{\texorpdfstring{Upper limit (about \(1 \mu \mathrm{~m}\)
):}{Upper limit (about 1 \textbackslash mu \textbackslash mathrm\{\textasciitilde m\} ):}}\label{upper-limit-about-1-mu-mathrmm}

\begin{itemize}
\tightlist
\item
  Balance of thermal and potential energy:
\end{itemize}

Every particle has thermal energy
\(\mathrm{E}_{\text {therm }}=(3 / 2) k_{B} T\) and potential energy
\(E_{\mathrm{pot}}=m g z=(4 \pi / 3) R^{3} \Delta \rho g z\), where
\(z\) is a characteristic length, e.g.~\(R\), and \(\Delta \rho\) is the
density difference between the particle and the dispersion medium. For a
significant influence of the thermal energy we need
\(E_{\mathrm{pot}}<E_{\text {therm }}\) and thus for a typical system
\(\left(\Delta \rho \approx 0.3 \mathrm{~g} \mathrm{~cm}^{-3}\right): \quad R<\left(9 k_{B} T / 8 \pi \Delta \rho g\right)^{1 / 4} \approx 1 \mu \mathrm{~m}\).

\subparagraph{Sedimentation:}\label{sedimentation}

Gravity will lead to sedimentation of the suspended particles (or
creaming, if the density of the particles is lower than the density of
the dispersion medium) and to the formation of a dense sediment.
\pandocbounded{\includegraphics[keepaspectratio]{soft-matter_files/mediabag/2025_01_27_a9af280ad12345678.jpg}}

The sedimentation velocity \(v_{\text {sed }}\) results from the balance
of gravity and hydrodynamic drag:

\[
\begin{aligned}
& \quad \text { gravity: } \\
F_{g}= & m_{\text {part }} g-m_{\text {liq }} g \\
= & V\left(\rho_{\text {part }}-\rho_{\text {liq }}\right) g \\
& =V \Delta \rho g \\
\downarrow & \text { sphere } \\
& F_{g}=\frac{4 \pi}{3} R^{3} \Delta \rho g
\end{aligned}
\]

\pandocbounded{\includegraphics[keepaspectratio]{soft-matter_files/mediabag/2025_01_27_a9af280ad123456789.jpg}}
friction (Stokes law): where \(m\) is the mass and \(\rho\) the density
of the particle (part) and liquid (liq), respectively, \(\Delta \rho\)
the density difference, \(V\) the volume and \(R\) the radius of the
particle, \(\eta\) the viscosity, \(v\) the sedimentation velocity and
\(g\) the acceleration due to gravity.

\subsubsection{Brownian motion and random
walks}\label{brownian-motion-and-random-walks}

Random collisions with the liquid molecules cause the particle to change
its direction of motion rapidly (resulting in `mixing' the sample). This
can be modelled as a random (drunk!) walk of \(N\) steps of length \(b\)
with a random angle between the individual steps. How far from the
origin does the particle get in N steps?
\pandocbounded{\includegraphics[keepaspectratio]{soft-matter_files/mediabag/2025_01_27_a9af280ad12345678910.jpg}}

Average over many tries (or many particles):

\[
\left\langle\underline{R}>=\sum_{j=1}^{N}\left\langle\underline{r}_{j}\right\rangle=0 \quad \text { since } \underline{r}_{j}\right. \text { can be in any direction }
\]

Mean square displacement:

\[
\begin{aligned}
&<\underline{R}^{2}>=<\underline{R} \cdot \underline{R}>=<\sum_{j=1}^{N} \underline{r}_{j} \cdot \sum_{k=1}^{N} \underline{r}_{k}>=\sum_{j=1}^{N} \sum_{k=1}^{N}\left\langle\underline{r}_{j} \cdot \underline{r}_{k}>\right. \\
& j=k:<\underline{r}_{j} \cdot \underline{r}_{k}>=\left\langle\underline{r}_{j}^{2}>=b^{2}\right. \\
& j \neq k:<\underline{r}_{j} \cdot \underline{r}_{k}>=b^{2}<\cos \theta>=0 \quad \text { since any } \theta \text { in } 0 \leqq \theta \leqq 2 \pi \\
&<\underline{R}^{2}>=\sum_{j=1}^{N} b^{2}=N b^{2} \propto t \quad \text { since time } t \propto N
\end{aligned}
\]

In fact a more careful treatment shows that the root mean-square
displacement \(<\underline{R}^{2}>1 / 2\)
\(=\sqrt{2 D t} \propto \sqrt{t}\) where for spherical particles the
diffusion constant \(D=k_{B} T / 6 \pi \eta R\), where \(R\) is the
particle radius.

Example: Consider uncharged protein spheres in water ( \(\rho\) protein
\(=1.35 \mathrm{~g} \mathrm{~cm}^{-3}\) ) at
\(20^{\circ} \mathrm{C}(\eta\) \(=1 \mathrm{Ps})\). We can calculate the
displacement due to Brownian motion and sedimentation respectively as a
function of the particle radius \(R\).

For Brownian motion, we have
\(\left\langle\underline{R}^{2}\right\rangle^{1 / 2}=\sqrt{2 D t}\) with
\(D=k_{B} T / 6 \pi \eta R\) For sedimentation, we have
\(v_{\text {sed }}=2 R^{2} \Delta \rho g / 9 \eta\) Hence we can make a
plot of displacement versus particle radius.

\begin{longtable}[]{@{}
  >{\centering\arraybackslash}p{(\linewidth - 8\tabcolsep) * \real{0.2000}}
  >{\centering\arraybackslash}p{(\linewidth - 8\tabcolsep) * \real{0.2000}}
  >{\centering\arraybackslash}p{(\linewidth - 8\tabcolsep) * \real{0.2000}}
  >{\centering\arraybackslash}p{(\linewidth - 8\tabcolsep) * \real{0.2000}}
  >{\centering\arraybackslash}p{(\linewidth - 8\tabcolsep) * \real{0.2000}}@{}}
\toprule\noalign{}
\begin{minipage}[b]{\linewidth}\centering
R
\end{minipage} & \begin{minipage}[b]{\linewidth}\centering
\(\mathrm{D}\left(20^{\circ} \mathrm{C}\right)\)
\end{minipage} & \begin{minipage}[b]{\linewidth}\centering
\(\left(\left\langle\underline{R}^{2}\right\rangle\right)^{1 / 2}\)
after 1 h
\end{minipage} & \begin{minipage}[b]{\linewidth}\centering
\end{minipage} & \begin{minipage}[b]{\linewidth}\centering
\(\mathrm{v}_{\text {sed }} \times 1 \mathrm{~h}\)
\end{minipage} \\
\midrule\noalign{}
\endhead
\bottomrule\noalign{}
\endlastfoot
1 nm & \(2.1 \cdot 10^{-10} \mathrm{~m}^{2} / \mathrm{s}\) &
\(1230 \mu \mathrm{~m}\) & \(>\) & 2.8 nm \\
10 nm & \(2.1 \cdot 10^{-11} \mathrm{~m}^{2} / \mathrm{s}\) &
\(390 \mu \mathrm{~m}\) & \(>\) & \(0.3 \mu \mathrm{~m}\) \\
100 nm & \(2.1 \cdot 10^{-12} \mathrm{~m}^{2} / \mathrm{s}\) &
\(123 \mu \mathrm{~m}\) & \(>\) & \(30 \mu \mathrm{~m}\) \\
\(1 \mu \mathrm{~m}\) &
\(2.1 \cdot 10^{-13} \mathrm{~m}^{2} / \mathrm{s}\) &
\(39 \mu \mathrm{~m}\) & \(<\) & 3 mm \\
\(10 \mu \mathrm{~m}\) &
\(2.1 \cdot 10^{-14} \mathrm{~m}^{2} / \mathrm{s}\) &
\(12 \mu \mathrm{~m}\) & \(<\) & 0.3 m \\
\end{longtable}

\pandocbounded{\includegraphics[keepaspectratio]{soft-matter_files/mediabag/2025_01_27_a9af280ad1234567891011.jpg}}

More generally we can write down equations which express the probability
of finding a particle at a given vector displacement from its starting
point. In the limit of a large number of random steps, one finds that in
1 dimension this probability is given by

\[
P_{N}(x) d x=\frac{1}{\sqrt{2 \pi<(\Delta x)^{2}>}} e^{-\frac{(x-\langle x\rangle)^{2}}{\left.2<(\Delta x)^{2}\right\rangle}} d x \quad \text { A 1d Gaussian distribution }
\]

This has mean \(\langle x\rangle=0\) and variance
\(\left\langle(\Delta x)^{2}\right\rangle \equiv \sigma^{2}>0\).

In 3 dimensions, the corresponding relation is

\[
P(\underline{r}) d \underline{r}=\frac{1}{\left(2 \pi \sigma^{2}\right)^{3 / 2}} e^{-r^{2} / 2 \sigma^{2}} d \underline{r}
\]

The distribution has its first moment at the origin and second moment

\[
\left\langle\underline{r}^{2}\right\rangle=\int_{-\infty}^{\infty} \underline{r}^{2} P(\underline{r}) d \underline{r}=\frac{1}{\left(2 \pi \sigma^{2}\right)^{3 / 2}} \int_{0}^{\pi} \sin \theta d \theta \int_{0}^{2 \pi} d \phi \int_{0}^{\infty} r^{4} e^{-r^{2} / 2 \sigma^{2}} d r=3 \sigma^{2}
\]

This latter result can also be obtained in a direct calculation,
considering \(N\) random steps in x direction, \(N\) in y direction and
\(N\) in z direction with
\(\left\langle x^{2}\right\rangle=\left\langle y^{2}\right\rangle=\left\langle z^{2}\right\rangle=\sigma^{2}\)
and thus
\(\left\langle\underline{r}^{2}\right\rangle=\left\langle x^{2}\right\rangle+\left\langle y^{2}\right\rangle+\left\langle z^{2}\right\rangle=3 \sigma^{2}=N b^{2}\)

\subsubsection{Large interface between dispersed phase and dispersion
medium}\label{large-interface-between-dispersed-phase-and-dispersion-medium}

In commercial terms it is often highly desirable to produce soft matter
systems whose (chemical) properties lie somewhere between those of bulk
and molecularly dispersed systems, i.e.~whose properties are not just
the sum of the contributions from the molecules in the bulk phases. To
achieve this, a significant proportion of the molecules should lie
within or close to the particle-medium interface, which is equivalent to
a large surface-to-volume ratio (`microheterogeneous' systems, link to
surface science).
\pandocbounded{\includegraphics[keepaspectratio]{soft-matter_files/mediabag/2025_01_27_a9af280ad123456789101112.jpg}}

Often colloidal particles have a spherical shape, which allows for a
relatively small surface-to-volume ratio, but they may also be
ellipsoids (prolate or oblate), discs, rods, fibres, random coils etc.

\subsubsection{Making measurements: Scattering
experiments}\label{making-measurements-scattering-experiments}

\pandocbounded{\includegraphics[keepaspectratio]{soft-matter_files/mediabag/2025_01_27_a9af280ad12345678910111213.jpg}}

The most commonly used experimental tool for studying soft matter
systems is scattering. The experiment may utilize light, neutrons or
X-rays as the scattering probe. Due to the random arrangement of
particles, a random diffraction pattern is observed. This fluctuates as
particles move in Brownian motion. The scattered intensity as measured
under a certain scattering angle will thus also fluctuate with time
(given the detector area is small enough).

\subsection{COLLOIDS}\label{colloids}

\subsubsection{Brownian Motion}\label{brownian-motion}

\subsubsection{Brief history:}\label{brief-history}

1828 Scottish botanist Robert Brown observed irregular motion of pollen
grains suspended in water. \textasciitilde1880 Many accepted that
Brownian motion was associated with the nature of heat, but no
quantitative theory 1905 Brownian motion explained by Albert Einstein
1926 Nobel Prize in Physics: Jean Baptist Perrin for investigations on
Brownian motion (sedimentation equilibrium and mean square displacement)
in connection with reality of molecules, linking Gas constant and
Avogadro's number which led to the acceptance of the `molecular
hypothesis'.

\subsubsection{Diffusion}\label{diffusion}

Idea: So far we considered a single particle and calculated the
probability for finding it after N steps in the volume element
\(d \underline{r}\) centred at \(\underline{r}\). We found that
\(\left\langle r^{2}\right\rangle=N b^{2}\), which suggests
\(\left\langle\underline{r}^{2}\right\rangle \propto t\).

Now we are interested in many particles, where each undergoes a random
walk and is not influenced by the others. We then ask: What fraction
will be in a certain volume element after time \(t\) ? This is the
process of diffusion, e.g.~spread of a drop of colour ink in water.

Mathematics: Suppose that the distribution of particles \(n(x, t)\) is
not uniform. This nonequilibrium situation will lead to a motion of
particles (diffusion) tending to increase entropy and restore a
homogeneous concentration. We are interested in the time-dependent
concentration (number per unit volume) profile of particles \(n(x, t)\).
In the case of an initially very narrow spatial distribution of
particles this will correspond to the probability of any particle having
moved to \(x\). The mean number of particles crossing unit area of a
plane (perpendicular to the \(x\)-axis) per unit time (i.e.~the `flux'),
is denoted by \(J_{x}\). This flux is expected to be (to a good
approximation) proportional to the concentration gradient:

\[
J(x, t)=-D \frac{\partial n(x, t)}{\partial x} \quad \text { Fick's law }
\]

where the diffusion constant (or coefficient) \(D\) has been defined as
the constant of proportionality

Now, conservation of particle number demands

\[
\begin{aligned}
& \frac{\partial}{\partial t}(n(x, t) A d x)=A J_{x}(x, t)-A J_{x}(x+d x, t) \\
& \text { Flux in ................. Flux out } \\
& =A\left(J_{x}(x, t)-\left(J_{x}(x, t)+\frac{\partial J_{x}(x, t)}{\partial x} d x\right)\right)
\end{aligned}
\]

and thus

\[
\frac{\partial n(x, t)}{\partial t}=-\frac{\partial J_{x}(x, t)}{\partial x} \quad \text { Particle conservation }
\]

Fick's law and particle conservation together give (assuming
\(D \neq D(x)\) ):

\[
\frac{\partial n(x, t)}{\partial t}=D \frac{\partial^{2} n(x, t)}{\partial x^{2}} \quad \text { Diffusion equation }
\]

The solution, up to a constant \(C\), which satisfies the initial
condition \((\mathrm{t}=0)\) of having all the particles close together
at \(x=0\) is

\[
n(x, t)=C \frac{1}{\sqrt{4 \pi D t}} \exp \left(-x^{2} / 4 D t\right) \quad \text { Gaussian Distribution }
\]

The constant \(C\) can be determined by
\(N_{\text {total }}=\int n d V=A \int n d x=A C\). Hence

\[
p(x, t) d x=\frac{A d x n(x, t)}{N_{\text {tot }}}=\frac{1}{\sqrt{4 \pi D t}} \exp \left(-x^{2} / 4 D t\right) d x
\]

\pandocbounded{\includegraphics[keepaspectratio]{soft-matter_files/mediabag/2025_01_27_a9af280ad1234567891011121314.jpg}}
\pandocbounded{\includegraphics[keepaspectratio]{soft-matter_files/mediabag/2025_01_27_a9af280ad123456789101112131415.jpg}}

But the spreading of the particle distribution with time is due to the
random walk of individual particles, for which we already have an
equation (1-D):

\[
p_{N}(x)=\frac{1}{\sqrt{2 \pi<(\Delta x)^{2}>}} \exp \left(-\frac{(x-<x>)^{2}}{2<(\Delta x)^{2}>}\right)
\]

Comparing the result from the two points of view and setting
\(\langle x\rangle=0\), we get \(<\Delta x^{2}>=2 D t\) As expected the
mean square displacement increases linearly with time, but we have now
found the constant of proportionality: \(2 D\). Generalisation to three
dimensions yields:

\[
\begin{array}{ll}
J=-D \nabla n(\underline{r}, t) & \text { Fick's law } \\
\frac{\partial n}{\partial t}=-\nabla \cdot \underline{J} & \text { Particle conservation } \\
\frac{\partial n}{\partial t}=D \nabla^{2} n & \text { Diffusion equation } \\
\left\langle\underline{r}^{2}\right\rangle=6 D t & \text { Mean square displacement }
\end{array}
\]

This last relation could constitute an alternative definition of the
diffusion coefficient, instead of Fick's law.

\subsubsection{Stokes-Einstein relation}\label{stokes-einstein-relation}

Suppose that the particles are subjected to an external force F in the x
direction, e.g.~gravity. In thermal equilibrium the Maxwell-Boltzmann
distribution is valid, i.e.~the particle density \(n(x)\) is given by

\[
n(x)=n_{0} \exp \left(-U(x) / k_{B} T\right)=n_{0} \exp \left(-F x / k_{B} T\right)
\]

where we assumed a constant force \(F\) in the last equation. In the
case of gravity \(F=m_{B} g\) with \(m_{B}\) the buoyant mass. \(n(x)\)
results from the balance between the motion of the particles due to the
external force setting up a concentration gradient, and the resultant
diffusion given by Fick's law.
\pandocbounded{\includegraphics[keepaspectratio]{soft-matter_files/mediabag/2025_01_27_a9af280ad12345678910111213141516.jpg}}

In the case of gravity, this leads to a sedimentation equilibrium. (a)
Flux due to external force, \(J_{F}\)

The velocity of a particle under an applied force \(F\) in a viscous
fluid can be written as \(v=\) \(F / \xi\) which defines the friction
coefficient \(\xi\). Hence

\[
J_{F}=n(x) v=\frac{n(x) F}{\xi}
\]

\begin{enumerate}
\def\labelenumi{(\alph{enumi})}
\setcounter{enumi}{1}
\tightlist
\item
  Diffusive flux, \(J_{D}\) \(J_{D}\) is given by Fick's Law (see
  above):
\end{enumerate}

\[
J_{D}(x)=-D \frac{\partial n(x)}{\partial x}
\]

Equating the two fluxes \(J_{F}=J_{D}\) we get

\[
\frac{n(x) F}{\xi}=-D \frac{\partial n(x)}{d x}=+D \frac{F}{k_{B} T} n(x)
\]

The second equation is obtained by differentiating the Maxwell-Boltzmann
distribution. This gives the relation between the diffusion and friction
coefficients:

\[
D=\frac{k_{B} T}{\xi}=\frac{k_{B} T}{6 \pi \eta R}
\]

The last equation applies to a spherical particle of radius \(R\) in a
fluid of viscosity \(\eta\), for which Stokes's Law gives
\(\xi=6 \pi \eta R\) (which applies only at low Reynold's number,
\(\rho R v / \eta\) \(\ll 1\) ) resulting in the Stokes-Einstein
relation.

\begin{itemize}
\tightlist
\item
  The Stokes-Einstein relation is a very deep result. It relates
  equilibrium fluctuations in a system to the energy dissipation when
  the system is driven off equilibrium. Here, the fluctuations in the
  fluid give rise to the diffusive motion of the suspended particle and
  \(D\) is therefore the `fluctuation' part. A sheared fluid will
  dissipate energy because of its finite viscosity and thus \(\eta\)
  represents the dissipative part.
\item
  More generally, Brownian motion sets a natural limit to the precision
  of physical measurements. Example: A mirror suspended on a torsion
  fibre reflects a spot of light onto a scale. The spot will jiggle due
  to the random impact of air molecules and the random motion of atoms
  in the quartz fibre. To reduce the jiggle, the apparatus has to be
  cooled. The relation between fluctuation and dissipation tells us
  where to cool. `This depends upon where {[}the mirror{]} is getting
  its 'kicks' from. If it is through the fibre, we cool it \ldots{} if
  the mirror is surrounded by a gas and is getting hit mostly by
  collisions in the gas, it is better to cool the gas. As a matter of
  fact, if we know where the damping of the oscillations comes from, it
  turns out that that is always the source of the fluctuations.'
  (Feynman, Chapter 41)
\end{itemize}

\subsubsection{Interacting colloidal
particles}\label{interacting-colloidal-particles}

So far we have dealt with colloidal particles as independent spheres
suspended in a medium. Now we will study concentrated suspensions in
which interparticle interactions cannot be ignored.

\paragraph{Samples: model hard-sphere
colloids}\label{samples-model-hard-sphere-colloids}

To be specific let us discuss a monodisperse colloid made up of
polymethyl- methacrylate (PMMA; perspex in bulk form) spheres, which are
covered with polymeric hairs (Poly-hydroxystearic acid) to stop them
sticking to each other. They show to a very good approximation hard
sphere behaviour.
\pandocbounded{\includegraphics[keepaspectratio]{soft-matter_files/mediabag/2025_01_27_a9af280ad1234567891011121314151617.jpg}}

The radius of the colloid is typically \(R \leq 0.5 \mu m\) with a very
small polydispersity (size variation) of \(\Delta R / R<0.05\). The
usual way of referring to particle densities is the `volume fraction'
\(\phi\), defined to be the fraction of the suspension occupied by
particles. If we have \(N\) monodisperse spheres with radius \(R\) in
the total volume \(V\), then

\[
\phi=\frac{V_{\mathrm{part}}}{V}=\frac{N}{V} \frac{4}{3} \pi R^{3}
\]

\paragraph{Phase Behaviour}\label{phase-behaviour}

As a reminder we first look at the interaction potential and phase
behaviour of a simple atomic substance, such as argon. A generic
inter-particle potential \(U(r)\) exhibits the infinite hard sphere
repulsion as well as an attractive part of depth \(\varepsilon\). The
\pandocbounded{\includegraphics[keepaspectratio]{soft-matter_files/mediabag/2025_01_27_a9af280ad123456789101112131415161718.jpg}}
\pandocbounded{\includegraphics[keepaspectratio]{soft-matter_files/mediabag/2025_01_27_a9af280ad12345678910111213141516171819.jpg}}
corresponding generic phase diagram in the temperature-density plane
shows the regions where gas, liquid, fluid and crystal exist. A rough
estimate of the critical temperature \(T_{C}\) is given by
\(k_{B} T_{C} \sim \varepsilon\). If \(\varepsilon \rightarrow 0\), as
in the case of hard spheres, we expect that there will be no critical
point and therefore no liquid phase. This is indeed the case.

\paragraph{Phase behaviour of hard-sphere
colloids:}\label{phase-behaviour-of-hard-sphere-colloids}

At very low densities, the structure of a hard-sphere suspension is like
that of a perfect gas; the positions of the colloidal particles are
practically uncorrelated and the chance of finding a particle in any
particular vicinity is almost independent of the presence of all the
other particles.

Upon an increase in particle density, their positions begin to become
correlated as the particles are mutually excluding. If we look inside a
suspension of volume fraction \(\phi \geq 0.3\), then we will find that,
at any one instant, each particle is surrounded by a `cage' of
neighbours (short-ranged order). The cage tightens as \(\phi\)
increases. The cage size
\pandocbounded{\includegraphics[keepaspectratio]{soft-matter_files/mediabag/2025_01_27_a9af280ad1234567891011121314151617181920.jpg}}
represents a dominant length scale in the system. Despite the appearance
of correlations between particle positions, a hard-sphere suspension
below \(\phi \lesssim 0.5\) is still in what one might call a `colloid
fluid' phase, i.e.~the arrangement of particles in such a suspension is
analogous to the arrangement of atoms in an atomic fluid (e.g.~fluid
argon) - on average the particle centres are more or
\pandocbounded{\includegraphics[keepaspectratio]{soft-matter_files/mediabag/2025_01_27_a9af280ad123456789101112131415161718192021.jpg}}
less randomly arranged, and each particle can, given time, wander
through the available volume. The structure is described by the radial
distribution function \(g(r)\). Given a particle centred at
\(\underline{r}=0\), \((N / V) g(\underline{r}) d \underline{r}\) is the
number of particles in \(d \underline{r}\) at a distance
\(\underline{r}\).

If we increase the volume fraction above \(\phi \approx 0.55\), however,
a phase transition occurs. The particles will spontaneously pack into a
crystalline arrangement. Particles now can only move round about their
average lattice site, but will not wander throughout the available
volume. This phase is called a `colloidal crystal'. Due to more local
freedom the crystal has a higher (!) entropy than the metastable fluid
from which it grows. It is an entropically-driven phase transition.
Between \(\phi=0.5\) and \(\phi=0.55\), we find coexistence between
colloidal fluid and crystal.
\pandocbounded{\includegraphics[keepaspectratio]{soft-matter_files/mediabag/2025_01_27_a9af280ad12345678910111213141516171819202122.jpg}}

Note that no colloidal liquid is formed ( \(T_{\mathcal{C}}=0\) ), as
expected for an interaction potential without attraction. At the highest
particle concentrations (above \(\phi \approx 0.58\) ) crystallization
may be avoided if the temperature is quenched very rapidly, and a
colloidal glass is formed. We will return to study the behaviour of
glasses in Part 4.

\subsubsection{Particle dynamics}\label{particle-dynamics}

\paragraph{Hard-spheres}\label{hard-spheres}

We are interested in the mean-square displacement
\(\left\langle\underline{r}^{2}(t)\right\rangle\) as a function of time
for different volume fractions. At low volume fractions, the particles
undergo Brownian motion (random-walk diffusion) due to collisions with
liquid molecules. The meansquare displacement
\(\left\langle\underline{r}^{2}(t)\right\rangle=6 D_{0} t\) as already
seen earlier. Above \(\phi \sim 0.3\), however, different regimes are
observed. At short times the particles diffuse with the short time
(self) diffusion constant \(D_{s}\). This is determined from the short
time limit and is smaller than the \(D_{0}\) measured for
\(\phi \rightarrow 0\). The motion of the particles (self diffusion) is
still driven by collisions with the liquid molecules, but in addition
the interactions between particles become significant. While the
particles are diffusing in their cages formed by their neighbours, the
hydrodynamic interaction
\pandocbounded{\includegraphics[keepaspectratio]{soft-matter_files/mediabag/2025_01_27_a9af280ad1234567891011121314151617181920212223.jpg}}
with the neighbours, transmitted through flows in the liquid, causes
slowing down relative to the free diffusion at low concentrations. At
intermediate times the particles encounter the neighbours and the
interactions slow the motion down. To make further progress, the
particle has to break out of the cage formed by its neighbours. Now the
particles experience a further interaction, direct interactions
(hard-sphere interactions), in addition to the hydrodynamic
interactions. The long-time and long-ranged movement is also diffusive,
i.e.~we still have
\(\left\langle\underline{r}^{2}(t)>\propto t\right.\), when the
particles undergo large-scale random-walk diffusion through many cages.
However, the motion is further slowed and a smaller diffusion constant
relative to the motion in the short time limit is observed, the long
time (self) diffusion constant \(D_{L}\).
\pandocbounded{\includegraphics[keepaspectratio]{soft-matter_files/mediabag/2025_01_27_a9af280ad123456789101112131415161718192021222324.jpg}}
short time: \(<r^{2}(t)>=6 D_{S} t\) short-time self diff. coeff.
\(D_{S}<D_{0}\) purely because of hydrodyn. interactions theoretical
predictions exist intermediate time: some theory long time:
\(<r^{2}(t)>=6 D_{L} t\) long-time self diff. coeff.
\(D_{L}<D_{S}\left(<D_{0}\right)\) because of hydrodyn. and direct
interactions theory very complicated, but some predictions exist

\subsection{POLYMERS}\label{polymers}

\subsubsection{Chemical structure}\label{chemical-structure}

A polymer is a large molecule made up of many small, simple chemical
units, joined together by chemical bonds. The basic unit of this
sequence is called a `monomer', and the number of units in the sequence
is called the `degree of polymerisation'. It is possible to have
polymers containing over \(10^{5}\) units and there are naturally
occurring polymers with a degree of polymerisation exceeding \(10^{9}\).
For example, polystyrene with a degree of polymerisation of \(10^{5}\)
has a molecular weight of about \(10^{7} \mathrm{~g} / \mathrm{mol}\)
and, if fully stretched out, would be about 25 \(\mu \mathrm{m}\) long.
Polymers play a central role in many fields, ranging from technology to
biology. This is reflected in a huge number of different chemical
structures. Given this manifold and the complexity of polymer molecules,
the theories are astonishingly simple. This is possible because of the
characteristic feature of polymers: The molecule itself is very large
and the macroscopic behaviour is dominated by this large-scale property
of the molecule. A description of the `fine structure' of the behaviour
of real polymers, however, is not attempted by (most of) these theories.
Polymers exist in very different architectures, such as linear,
branched, star or cross-linked (see Chapter 1). Furthermore, a large
variation in the chemical structure may be achieved by combining
different monomers (copolymerisation). Then for the coupling of two
monomeric units in the chain two limiting cases exist; statistical or
block structure. We will focus on linear homopolymers, i.e.~no branch
points and all subunits are the same.

\subsubsection{Examples:}\label{examples}

\begin{longtable}[]{@{}
  >{\centering\arraybackslash}p{(\linewidth - 2\tabcolsep) * \real{0.5000}}
  >{\centering\arraybackslash}p{(\linewidth - 2\tabcolsep) * \real{0.5000}}@{}}
\toprule\noalign{}
\begin{minipage}[b]{\linewidth}\centering
Polyethylene (PE)
\end{minipage} & \begin{minipage}[b]{\linewidth}\centering
Poly(methyl methacrylate) (PMMA)
\end{minipage} \\
\midrule\noalign{}
\endhead
\bottomrule\noalign{}
\endlastfoot
- \(\left[\mathrm{CH}_{2}\right]\) - &
COC(=O)C(C)({[}14CH3{]}){[}SiH3{]} \\
\end{longtable}

Polystyrene (PS) Natural rubber methacrylate) (PMMA)

\begin{itemize}
\tightlist
\item
  \(\left[\mathrm{CH}_{2}-\mathrm{CH}\right]-\)
\end{itemize}

\subsubsection{Polymer conformations}\label{polymer-conformations}

\paragraph{Models for the conformation of
polymers}\label{models-for-the-conformation-of-polymers}

In order to understand the properties of most substances, we must
consider a large assembly of molecules. In the case of polymers,
however, the molecules themselves are very large and due to their
flexibility they can take up an enormous number of configurations by
rotation of chemical bonds. The shape of the polymer can therefore only
be usefully described statistically and one need to use statistical
mechanics to calculate the characteristics of even an isolated polymer.
To be able to investigate the properties of a single polymer and to
neglect interactions between polymers, the polymer is placed in a very
dilute solution. In this chapter, we will theoretically investigate the
properties of an isolated, single polymer chain in solution (which in
addition is linear and consists of only one kind of monomers).

\paragraph{Freely-jointed chain}\label{freely-jointed-chain}

Many polymers are highly flexible and are coiled up in solution. In a
simple model we thus describe a polymer as consisting of a large number
of segments freely joined up, where all angles between segments are
assumed to be equally likely.

The momomers are located at positions \(\underline{R}_{\mathrm{j}}\) and
connected by bonds
\(\underline{r}_{j}=\underline{R}_{j}-\underline{R}_{j-1}\) of length
\(\left|\underline{r}_{j}\right|=b_{0}\). The end-to-end vector is
\(\left|\underline{R}_{N}-\underline{R}_{0}\right|=\sum_{j=1}^{N} r_{j}\).
At any instant, the configuration (arrangement) of the polymer is one
realisation of an N -step random walk in three dimensions. In time, the
various segments undergo Brownian motion and the
\pandocbounded{\includegraphics[keepaspectratio]{soft-matter_files/mediabag/2025_01_27_a9af280ad12345678910111213141516171819202122232425.jpg}}
polymer fluctuates between all possible configurations of the random
walk
\pandocbounded{\includegraphics[keepaspectratio]{soft-matter_files/mediabag/2025_01_27_a9af280ad1234567891011121314151617181920212223242526.jpg}}

Size and shape of a freely-jointed chain: Based on the end-to-end vector
\(\underline{R}\), the size and shape of the chain can thus be described
probabilistically:

\[
\begin{aligned}
& <\underline{R}>=\sum_{j=1}^{N}\left\langle\underline{r}_{j}\right\rangle \\
& <\underline{R}^{2}>=\sum_{j=1}^{N} \sum_{k=1}^{N}<\underline{r}_{j} \underline{\underline{L}}_{k}>=N b_{0}^{2} \\
& P(\underline{R}) d \underline{R}=\left(\frac{3}{2 \pi<\underline{R}^{2}>}\right)^{3 / 2} e^{-\frac{3 \underline{R}^{2}}{2<\underline{R}^{2}>}} d R \quad \text { for large } \mathrm{N} \text { (Gaussian distribution) }
\end{aligned}
\]

While the end-to-end distance represents a well-defined quantity for a
linear chain, we need a more versatile measure of the size for more
complicated architectures, such as branched or star-shaped polymers.
This is provided by the mean square distance from the centre of mass
\(\underline{R}_{\mathrm{G}}\), the so-called radius of gyration
\(R_{g}\), which is also a measure of the extension of a (random) chain:

\[
\begin{aligned}
& \underline{R}_{G}=\frac{1}{N} \sum_{j=1}^{N} \underline{R}_{j} \\
& R_{g}^{2}=\frac{1}{N} \sum_{j=1}^{N}\left(\underline{R}_{j}-\underline{R}_{G}\right)^{2}=\frac{1}{2 N^{2}} \sum_{j=1}^{N} \sum_{k=1}^{N}\left(\underline{R}_{j}-\underline{R}_{k}\right)^{2} \quad \quad \text { (see problem) } \\
& \therefore\left\langle R_{g}^{2}\right\rangle=\frac{1}{2 N^{2}} \sum_{j=1}^{N} \sum_{k=1}^{N}<\left(\underline{R}_{j}-\underline{R}_{k}\right)^{2}>=\frac{b_{0^{2}}}{2 N^{2}} \sum_{j=1}^{N} \sum_{k=1}^{N}|j-k|
\end{aligned}
\]

Now transform to continuous variables
\pandocbounded{\includegraphics[keepaspectratio]{soft-matter_files/mediabag/2025_01_27_a9af280ad123456789101112131415161718192021222324252627.jpg}}

Thus we find the simple result that the mean squared gyration radius is
proportional to the mean squared end-to-end distance.

\paragraph{Freely-rotating chain}\label{freely-rotating-chain}

In a polymer molecule the bond angles are usually restricted, which
leads to a limited flexibility of the molecule. Let us consider the case
of \(n\)-butane:

\[
\mathrm{H}_{3} \mathrm{C}-\mathrm{CH}_{2}-\mathrm{CH}_{2}-\mathrm{CH}_{3}
\]

While the \(\mathrm{C}-\mathrm{C}\) bond angle is fixed at about
\(120^{\circ}\), rotation about the bond is possible.

Nevertheless, the potential energy of a configuration depends on the
valence angle:
\pandocbounded{\includegraphics[keepaspectratio]{soft-matter_files/mediabag/2025_01_27_a9af280ad12345678910111213141516171819202122232425262728.jpg}}

At low temperatures ( \(k_{B} T\) \textless config. energy) the
configuration will thus be predominantly trans. As the temperature is
increased ( \(k_{B} T \sim\) config. energy), there will also be gauche
configurations and at high temperatures ( \(k_{B} T \gg\) config.
energy), any angle will be possible. This suggests that a model should
be based on fixed angles between bonds, but free rotation about the
bonds. This model is called the freely-rotating chain.

We start with a fixed configuration of \(\underline{r}_{l}\),
\(\underline{r}_{2}, \ldots, \underline{r}_{j-1}\) and then add the next
segment \(\underline{r}_{j}\). While the bond angle \(\Theta\) is given
by the chemistry of the molecule, the segment can still freely rotate
about the axis defined by \(r_{j-1}\), i.e.~\(\varphi\) can take any
value \(0 \leq \varphi \leq 2 \pi\). If we average \(\underline{r}_{j}\)
over \(\varphi\), while keeping
\(\underline{r}_{1}, \underline{r}_{2}, \ldots, \underline{r}_{j-1}\)
fixed, only the component in \(\underline{r}_{j}\) direction remains:
\(\left.<\underline{r}_{j}\right\rangle_{\underline{r}}, \underline{r}_{2}, \ldots, \underline{r}_{j-1}\)
fixed \(=\cos \Theta \underline{r}_{j-1}\)

For the calculation of \(\left\langle R^{2}\right\rangle\) we also need
\(\left\langle\underline{r}_{j} \cdot \underline{r}_{k}\right\rangle\),
\pandocbounded{\includegraphics[keepaspectratio]{soft-matter_files/mediabag/2025_01_27_a9af280ad1234567891011121314151617181920212223242526272829.jpg}}
which we obtain by multiplying both sides with \(\underline{r}_{k}\) and
taking the average:
\(\left\langle r_{j} \underline{x}_{k}\right\rangle=\cos \Theta\left\langle r_{j-1} \underline{x}_{k}\right\rangle=\cos ^{2} \Theta\left\langle r_{j-2} \underline{\Sigma}_{k}\right\rangle=\ldots=\left(\cos ^{|j-k|} \Theta\right)\left\langle r_{k}{\underline{x_{k}}}\right\rangle=\left(\cos ^{|j-k|} \Theta\right) b_{0}^{2}\)
Since \(\cos \Theta<1\), correlations between \(\underline{r}_{j}\) and
\(\underline{r}_{k}\) decrease with increasing distance \(|j-k|\)
between the links and the orientations of distant links become
uncorrelated. The end-to-end distance
\(\left\langle R^{2}\right\rangle\) of a freely-rotating chain is hence

\[
\begin{aligned}
\left\langle R^{2}\right\rangle & =\sum_{j=1}^{N} \sum_{k=1}^{N}\left\langle\underline{r}_{j} \cdot \underline{r}_{k}\right\rangle=b_{0}^{2} \sum_{j=1}^{N} \sum_{k=1}^{N}\left(\cos ^{(j-k \mid} \Theta\right)=N b_{0}^{2}+2 b_{0}^{2} \sum_{j=1}^{N} \sum_{k=1}^{j-1} \cos ^{(j-k)} \Theta \\
& =N b_{0}^{2}+2 b_{0}^{2} \sum_{j=1}^{N} \cos ^{j} \Theta \sum_{k=1}^{j-1} \cos ^{-k} \Theta
\end{aligned}
\]

where we used the same argument as above to deal with \(|j-k|\). To
calculate these two sums we consider the geometric progression:

\[
\begin{aligned}
& S=\sum_{m=1}^{M} x^{m}=x+x^{2}+\ldots .+x^{M} \quad \therefore x S=x^{2}+x^{3}+\ldots .+x^{M+1} \quad \therefore S-x S=x-x^{M+1} \\
& \therefore S=\frac{x-x^{M+1}}{1-x}
\end{aligned}
\]

With the help of this formula we get

\[
\begin{aligned}
\left\langle R^{2}\right\rangle & =N b_{0}^{2}+2 b_{0}^{2} \sum_{j=1}^{N} \cos ^{j} \Theta \frac{\frac{1}{\cos \Theta}-\frac{1}{\cos ^{j} \Theta}}{1-\frac{1}{\cos \Theta}}=N b_{0}^{2}+\frac{2 b_{0}^{2}}{\cos \Theta-1}\left(\sum_{j=1}^{N} \cos ^{j} \Theta-\sum_{j=1}^{N} \cos \Theta\right) \\
& =N b_{0}^{2}+\frac{2 b_{0}^{2}}{\cos \Theta-1}\left(\frac{\cos \Theta-\cos ^{N+1} \Theta}{1-\cos \Theta}-N \cos \Theta\right)
\end{aligned}
\]

For large N this can be simplified
\(\left\langle R^{2}\right\rangle \approx N b_{0}^{2}+\frac{2 b_{0}^{2}}{1-\cos \Theta} N \cos \Theta=N b_{0}^{2}\left(\frac{1+\cos \Theta}{1-\cos \Theta}\right)=C N b_{0}^{2}\)
with \(C=(1+\cos \Theta) /(1-\cos \Theta)\). To get a better idea of the
effect of a fixed angle \(\Theta\), i.e.~going from a freely-jointed to
a freely-rotating chain, we look at a few special (but not necessarily
very realistic) cases: (1)
\(\Theta \rightarrow 0 \Rightarrow \cos \Theta \rightarrow 1-\frac{\Theta^{2}}{2} \Rightarrow C=\frac{2-\Theta^{2} / 2}{\Theta^{2} / 2} \approx \frac{4}{\Theta^{2}} \quad\left(\mathrm{C} \approx 500\right.\)
for \(\left.\Theta=5^{0}\right)\)

\[
\therefore\left\langle R^{2}\right\rangle \gg N b_{0}^{2}
\]

\pandocbounded{\includegraphics[keepaspectratio]{soft-matter_files/mediabag/2025_01_27_a9af280ad123456789101112131415161718192021222324252627282930.jpg}}
(2)

\[
\begin{aligned}
\Theta \rightarrow \pi-\delta \quad & \Rightarrow \cos \Theta \rightarrow-1-\frac{\delta^{2}}{2} \Rightarrow C=\frac{\delta^{2} / 2}{2-\delta^{2} / 2} \approx \frac{\delta^{2}}{4} \quad\left(\mathrm{C} \approx 2 \times 10^{-3} \text { for } \Theta=175^{0}\right) \\
& \therefore\left\langle R^{2}\right\rangle \ll N b_{0}^{2}
\end{aligned}
\]

\pandocbounded{\includegraphics[keepaspectratio]{soft-matter_files/mediabag/2025_01_27_a9af280ad12345678910111213141516171819202122232425262728293031.jpg}}
(3)
\(\Theta \rightarrow \pi / 2 \Rightarrow \cos \Theta \rightarrow 0 \Rightarrow C=1\)

\[
\therefore\left\langle R^{2}\right\rangle=N b_{0}^{2}
\]

\pandocbounded{\includegraphics[keepaspectratio]{soft-matter_files/mediabag/2025_01_27_a9af280ad1234567891011121314151617181920212223242526272829303132.jpg}}

Again, we get \(\left\langle R^{2}\right\rangle \propto N\), which
suggests that a long freely-rotating chain can be represented by an
equivalent freely-jointed chain with \(N\) 'segments of length \(b\).
Real and effective chain must have the same actual length (
\(N b_{0}=N^{\prime} b\) ) and the same end-to-end distance \(<R^{2}>\)
i.e.~\(\left(C N b_{0}^{2}=C N^{\prime} b^{2}\right)\). These
constraints result in \(b=C b_{0}\) and \(N^{\prime}=N / C\). This has
important consequences:

\begin{itemize}
\tightlist
\item
  All sufficiently long flexible chains have identical • \(b\) is the so
  called ``Kuhn'' behaviour as regards their dimensions: the chemical
  details are hidden in \(N^{\prime}\) and \(b\).
\item
  While individual monomer pairs are not totally statistical segment
  length (and twice the ``persistence length'' \(l_{p}\) ) flexible,
  groups of monomers are
\item
  \(C\) represents the number of monomers over which the orientational
  correlation is lost
  \pandocbounded{\includegraphics[keepaspectratio]{soft-matter_files/mediabag/2025_01_27_a9af280ad123456789101112131415161718192021222324252627282930313233.jpg}}
\end{itemize}

\paragraph{Excluded volume effects}\label{excluded-volume-effects}

The fact that two monomers cannot occupy the same space has consequences
on different length scales. On a local length scale this prevents
neighbouring monomers from coming too close together. This effect is
taken into account in terms of a restricted bond-angles, which prevents
them from overlaping. Non-overlap, i.e.~excluded volume, of distant
monomers along the chain has also to be taken into account and can have
surprisingly large effects.

To estimate the importance of this effect, we consider the fraction of
coil volume actually occupied by monomers: volume actually occupied by
monomers: \(\quad V_{N}=N V_{1} \sim N b^{3}\) (where \(\mathrm{V}_{1}\)
is the volume of a monomer) volume occupied by the whole coil:

\[
V_{\text {coil }}=\frac{4 \pi}{3}<R_{g}^{2}>^{3 / 2} \sim \frac{4 \pi}{3} N^{3 / 2} b^{3}
\]

\(\therefore \frac{V_{N}}{V_{\text {coil }}}=\frac{N b^{3}}{(4 \pi / 3) N^{3 / 2} b^{3}} \sim N^{-1 / 2}\)
e.g.~\(N=10^{4}\) monomers occupy only about \(1 \%\) of the whole coil
volume.

The overall chain size \(<R_{g}{ }^{2>1 / 2}\) is determined by the
competition of two effects. Entropy (and chain connectivity) favour a
compact chain and avoid the more unlikely stretched configurations,
while repulsive excluded volume interactions want to expand the chain to
avoid overlap. Based on this balance we will `calculate' the effect of
excluded volume in a very hand-waving way. (Due to a fortuitous
cancellation of errors introduced by various approximations, the result
is practically identical to more rigorous treatments, which are very
involved.) We consider the Helmholtz function of a single chain, which
is regarded as an assembly of particles with constant volume
\(\mathrm{NV}_{1}\) at constant temperature T :

\[
F=U-T S
\]

The entropy S is given by

\[
S=k_{B} \ln (\mathrm{no} \text {. of configurations })
\]

where for a given \(\underline{R}\) the number of configurations is
expected to be proportional to

\[
P(\underline{R})=\left(\frac{3}{2 \pi<R^{2}>}\right)^{3 / 2} e^{-\frac{3 R^{2}}{2<R^{2}>}}
\]

and hence

\[
S \sim \frac{-3 k_{B} R^{2}}{2 N b^{2}}+\text { terms indep. of } \mathrm{R}
\]

The internal energy \(U\) includes the kinetic and potential energy.
However, the kinetic energy is independent of the configuration and thus
of \(\underline{R}\) and we only have to consider the potential energy.
To estimate the potential energy, we disregard the connectivity of the
chain and calculate the interaction energy of a `segment gas' confined
in a volume \(R^{3}\). The probability of a monomer to lie in this
volume is given by the fraction of total coil volume occupied by
monomers, which we estimated above to be \(N V_{1} / R^{3}\)

Thus the probability of monomer-monomer contacts is
\(N^{2} V_{1} / R^{3} \sim N^{1 / 2}\). With an energy \(\varepsilon\)
of a monomer-monomer contact, the potential energy
\(U \sim \varepsilon N^{2} V_{1} / R^{3}\). We thus obtain

\[
F=\frac{\varepsilon N^{2} V_{1}}{R^{3}}+\frac{3 k_{B} T R^{2}}{2 N b^{2}}+\text { terms indep. of } \mathrm{R}
\]

which can be minimized with respect to \(R\), i.e.~\(d F / d R=0\),
yielding

\[
R^{5}=\frac{\varepsilon V_{1} b^{2}}{k_{B} T} N^{3} \sim \frac{\varepsilon}{k_{B} T} N^{3} b^{5} \quad \therefore R \sim N^{3 / 5} b
\]

Simulations give a very similar scaling, \(\mathrm{R} \sim N^{0.588}\).
The chain can no longer be modelled by a random walk, but has to be
described by a self-avoiding random walk. The distribution of end-to-end
distances is also not Gaussian. Although the difference between an
exponent of 0.5 (as is characteristic for the freelyjointed and
freely-rotating chains, i.e.~a random walk) and 0.6 (excluded volume
chain, i.e.~self-avoiding random walk) seems small, it has a large
effect at large N . For example, for \(N=10^{4}, R=N^{0.5} b=100 b\),
while \(R=N^{0.6} b=251 b\), which corresponds to a swelling of the
chain by a factor of 2.5 .

\subsubsection{Good, poor and theta
solvents}\label{good-poor-and-theta-solvents}

So far we only considered monomer-monomer interactions, which we assumed
are purely repulsive, and neglected the influence of the solvent.
However, the type of solvent has a great effect on the polymer size. If
there is a high affinity with the solvent (`good solvent') the polymer
swells, while it will shrink in a `poor solvent'. We consider a lattice
model, where each lattice site has \(z\) nearest neighbours and there
are \(N_{s}\) solvent molecules, \(N_{p}\) monomers energies of
interaction are \(\varepsilon_{s s}\) for and
\(\mathrm{N}_{\mathrm{sp}}\) solvent-monomer contacts. The
solvent-solvent, \(\varepsilon_{\mathrm{pp}}\) for monomer- monomer and
\(\varepsilon_{s p}\) for solvent-monomer interactions. Then the energy
of mixing \(\Delta U_{\operatorname{mix}}\) is given by

\[
\Delta U_{\operatorname{mix}}=U-\left(U_{S}+U_{p}\right)
\]

where energy of pure solvent
\(U_{s}=\frac{z N_{s} \varepsilon_{s s}}{2}\) energy of pure polymer
\(U_{p}=\frac{z N_{p} \varepsilon_{p p}}{2}\)
\pandocbounded{\includegraphics[keepaspectratio]{soft-matter_files/mediabag/2025_01_27_a9af280ad12345678910111213141516171819202122232425262728293031323334.jpg}}
energy of solution
\(U=N_{s p} \varepsilon_{s p}+\frac{\left(z N_{s}-N_{s p}\right) \varepsilon_{s s}}{2}+\frac{\left(z N_{p}-N_{s p}\right) \varepsilon_{p p}}{2}\)
Hence we obtain for the energy of mixing

\[
\Delta U_{\mathrm{mix}}=N_{s p}\left[\varepsilon_{s p}-\frac{1}{2}\left(\varepsilon_{s s}+\varepsilon_{p p}\right)\right]
\]

which can be either positive or negative:

\begin{itemize}
\tightlist
\item
  \(\varepsilon_{s p}<\frac{1}{2}\left(\varepsilon_{s s}+\varepsilon_{p p}\right) \quad \therefore \Delta \mathrm{U}_{\text {mix }}<0\)
  \pandocbounded{\includegraphics[keepaspectratio]{soft-matter_files/mediabag/2025_01_27_a9af280ad1234567891011121314151617181920212223242526272829303132333435.jpg}}
\end{itemize}

This is the case of a `good solvent', because the monomers prefer to be
near the solvent molecules. Excluded volume effects then expand the
chain.

\begin{itemize}
\tightlist
\item
  \(\varepsilon_{s p}>\frac{1}{2}\left(\varepsilon_{s s}+\varepsilon_{p p}\right) \quad \therefore \Delta \mathrm{U}_{\text {mix }}>0\)
\end{itemize}

This is the case of a `poor solvent', because the monomers prefer to be
near to each other (and similarly for the solvent molecules). The
attraction between the different monomers offset the excluded volume
effect. The importance of the attractions generally depends on
temperature. At very high temperatures the coil is expanded and the
solvent quality is good. In contrast, at very low temperatures, the
solvent quality is poor, attraction dominates, the coil collapses and
phase separation is observed. In between these two limits, there is a
temperature, the so-called theta temperature \(\Theta\), where the coil
has ideal dimensions and the effects of excluded volume and attraction
cancel each other. The solvent at \(T=\Theta\) is called a `theta
solvent'. The stronger the attractions the higher \(\Theta\) will be,
while for weak attractions \(\Theta\) is low. A full treatment of the
coil expansion is rather involved and has to take into account excluded
volume, attractions, configurational entropy and entropy of mixing.

\subsubsection{Summary}\label{summary}

A polymer has characteristics on different length scales. On a very
global length scale, it has a molar mass \(M\) and an overall size which
can be characterised by the root mean square end-to-end distance
\(<R^{2}>1 / 2\) or radius of gyration
\(<R g^{2}>1 / 2 \propto N^{v} \propto M^{v}\), where \(v=1 / 2\) for a
freely-jointed or freely-rotating chain (random walk) and \(v=3 / 5\)
for an excluded volume chain (self-avoiding random walk). On a smaller
length scale the behaviour will be dominated by the finite flexibility
or `persistence' of the chain, which is characterised by the Kuhn length
b. The chain will essentially behave like a stiff rod on this length
scale. This rod typically has a constant mass per length, \(M / L\), and
thus \(M \alpha L\). Finally, the local cross-sectional structure will
be observed on an even smaller length scale.

\subsubsection{Concentrated polymer
solutions}\label{concentrated-polymer-solutions}

Previously we considered a single polymer in a very dilute solution. Now
we increase the concentration in steps until we reach bulk polymers. A
special aspect of bulk polymers will then be discussed in more detail in
the following chapter. The most important regimes of concentration are:
(A) Dilute:
\pandocbounded{\includegraphics[keepaspectratio]{soft-matter_files/mediabag/2025_01_27_a9af280ad123456789101112131415161718192021222324252627282930313233343536.jpg}}

The polymer coils are well-separated on average. `Dilute' means:
\pandocbounded{\includegraphics[keepaspectratio]{soft-matter_files/mediabag/2025_01_27_a9af280ad12345678910111213141516171819202122232425262728293031323334353637.jpg}}
(B) Overlap concentration c*:
\pandocbounded{\includegraphics[keepaspectratio]{soft-matter_files/mediabag/2025_01_27_a9af280ad1234567891011121314151617181920212223242526272829303132333435363738.jpg}}

Overlap occurs when the volume fraction of coils reaches unity and thus

\[
\frac{c^{*}}{M} N_{A} \frac{4 \pi}{3} R_{g}^{3} \sim 1 \quad \therefore c^{*}=\frac{3 M}{4 \pi N_{A} R_{g}^{3}}
\]

using \(R g=<R g^{2>^{1 / 2}}=B M^{\nu}\) gives

\[
c^{*}=\frac{3}{4 \pi N_{A} B^{3}} M^{1-3 v}
\]

Example: Polystyrene with \(M=10^{6} \mathrm{~g} \mathrm{~mol}^{-1}\) in
a good solvent ( \(v=0.6\) ) and
\(B=0.028 \mathrm{~nm}\left(\mathrm{~g} \mathrm{~mol}^{-1}\right)^{-0.6}\)
leads to
\(\mathrm{c}^{*}=0.29 \mathrm{~kg} \mathrm{~m}^{-3}=0.29 \mathrm{mg} / \mathrm{ml}\).
With the density of polystyrene
\(\rho=1050 \mathrm{~kg} \mathrm{~m}^{-3}\), the volume fraction of
monomers is
\(\mathrm{c}^{*} / \rho=0.28 \times 10^{-3} . \mathrm{c}^{*}\) can be
very small for large polymers. (C) Semi-dilute:
\pandocbounded{\includegraphics[keepaspectratio]{soft-matter_files/mediabag/2025_01_27_a9af280ad123456789101112131415161718192021222324252627282930313233343536373839.jpg}}

\paragraph{Abstract}\label{abstract}

The concentration is larger than the overlap concentration
\(\mathrm{c}^{*}\), but still much smaller than the bulk density. The
coils interpenetrate and entangle, but the solution is still mostly
solvent

\paragraph{Concentrated:}\label{concentrated}

In this case the concentration is very close to the bulk density and the
polymer monomers occupy a significant fraction of the total volume.

\paragraph{Bulk polymers:}\label{bulk-polymers}

Bulk polymers are divided into two main classes, characterised by
whether they are cross-linked or not. There are elastomers or rubbers
with a low degree of cross-linking and thermosets with a high degree of
cross-linking. We will investigate the behaviour of rubbers in the next
chapter. The second class are thermoplastics, which are not
cross-linked. Most everyday plastic products are thermoplastics. We will
briefly discuss their behaviour upon cooling, which shows similarities
to the behaviour of colloids. At high temperature the free energy is
dominated by the entropic terms. The melt resembles a random assembly of
mobile, intertwined, flexible coils with a density similar to the
density of the corresponding monomer liquid. Upon cooling the potential
energy takes over and the bonds are restricted in their rotation leading
to configurations which are more straightened out. Below the melting
temperature \(\mathrm{T}_{\mathrm{m}}\), a crystal is the lowest free
energy state. Crystallisation, however, requires significant ordering of
the initially random melt and is only possible if cooling occurs slow
enough. If the melt is rapidly cooled below the glass transition
temperature \(T_{g}\left(<T_{m}\right)\), then instead of a crystal a
glass is formed, which represents an amorphous metastable, but
long-lived state. Although the polymers can still vibrate, they can no
longer move. Solid thermoplastics are frequently a mixture of
crystalline and amorphous structures.

\subsection{SURFACTANTS}\label{surfactants}

\subsubsection{Structure and examples}\label{structure-and-examples}

In this chapter we discuss the behaviour of a special class of
molecules. In these molecules one end contains a `hydrophilic'
(water-loving) part, while the other end is `hydrophobic'
(water-hating). They are called `amphiphilic' (loving both) molecules,
which reflects their structure, or `surfactants' (from SURFace ACTive
AgeNT), which refers to their behaviour in solution.
\pandocbounded{\includegraphics[keepaspectratio]{soft-matter_files/mediabag/2025_01_27_a9af280ad12345678910111213141516171819202122232425262728293031323334353637383940.jpg}}

A few chemical structures are shown below. They illustrates that the
hydrophobic tail usually consits of hydrocarbon chains of different
lengths and that the hydrophilic head might either be positively or
negatively charged, zwitterionic or uncharged.
\pandocbounded{\includegraphics[keepaspectratio]{soft-matter_files/mediabag/2025_01_27_a9af280ad1234567891011121314151617181920212223242526272829303132333435363738394041.jpg}}

The hydrocarbon chains are insoluble in water. The molecules are thus
preferentially located at the surface, which allows the hydrophilic head
to be surrounded by water and the hydrophobic chains to avoid contact
with water. There is always an equilibrium between surfactants at the
surface and in the bulk of the solution. The coverage of the surface
leads to a reduction of the surface tension with increasing surfactant
concentration.

\subsubsection{Self-assembled
structures}\label{self-assembled-structures}

\paragraph{Introduction}\label{introduction-1}

Above a so-called critical micellar concentration (cmc) surfactants
self-assemble in solution spontaneously into larger structures. (In the
following we will consider aqueous solutions, although the arguments
also apply to other polar or non-polar (organic) solvents.) This allows
the hydrophobic parts to crowd together while
\pandocbounded{\includegraphics[keepaspectratio]{soft-matter_files/mediabag/2025_01_27_a9af280ad123456789101112131415161718192021222324252627282930313233343536373839404142.jpg}}
being `shielded' by the hydrophilic heads. The density of the
hydrophobic cores is very similar to the density of fluid hydro-carbons
and the random arrangements of the chains resemble closely a fluid
structure.

The surfactant assemblies are not held together by chemical bonds, but
only by weak interactions ( \(\lesssim k B T\) ). Their existence and
properties are thus determined by a delicate balance between different
effects, such as the transfer of hydrophobic chains into the core,
interactions between the head group and the entropy of mixing. Small
changes in control parameters, for example temperature, salt
concentration or pH , thus have large effects on the characteristics of
the surfactant aggregates. Nevertheless, for given conditions, they have
very well-defined properties (shape, size etc.).

\paragraph{Shape of surfactant
assemblies}\label{shape-of-surfactant-assemblies}

Surfactants spontaneously self-assemble into a variety of different
structures. We use geometric or `packing' considerations to understand
and predict the shape of surfactant aggregates. In this model the
geometry of a surfactant molecule is described using the following
parameters:

\begin{itemize}
\tightlist
\item
  optimal headgroup area \(a_{0}\) : As discussed in the previous
  section, this depends on a delicate balance of forces and is thus not
  only controlled by the `chemistry' of the surfactant molecule, but
  also depends on different control parameters of the solution, such as
  salt concentration, pH or temperature.
\item
  volume v of the hydrophobic part: The hydrophobic part usually
  consists of hydrocarbon chains and for saturated hydrocarbons the
  volume \(v\) can be approximated by
  \(v \approx(27.4+26.9 \mathrm{n}) \times 10^{-3} \mathrm{~nm}^{3}\)
  where n is the number of carbon atoms.
\item
  critical chain length \(\boldsymbol{l}_{\boldsymbol{c}}\) : The
  maximum effective length of the hydrophobic chains is called the
  critical chain length \(l_{c}\), which has to be shorter than the
  fully extended molecular length of the chain \(l_{\max }\). For
  saturated hydrocarbons the critical length can be estimated using
  \(l_{c} \leq l_{\max } \approx(0.154+0.1265 \mathrm{n}) \mathrm{nm}\)
  The critical chain length heavily depends on the `chemistry' of the
  molecule, for example on the presence of double bonds or branching, as
  well as the temperature.
\end{itemize}

The structure which will be adopted is determined by a balance between
entropy, which favours small aggregates, and energy considerations: A
certain shape or size might only be possible by imposing a headgroup
area \(a>a_{0}\), which is energetically not favourable. We will now
establish the criteria for the different shapes. (A) Spherical micelle:
\pandocbounded{\includegraphics[keepaspectratio]{soft-matter_files/mediabag/2025_01_27_a9af280ad12345678910111213141516171819202122232425262728293031323334353637383940414243.jpg}}

For a spherical micelle with aggregation number N , the total volume and
surface area are given by

\[
\begin{gathered}
N v=\frac{4 \pi}{3} R^{3} \\
N a_{0}=4 \pi R^{2} \\
\therefore \frac{v}{a_{0}}=\frac{R}{3}<\frac{l_{c}}{3}
\end{gathered}
\]

where we used the fact that the radius \(R\) cannot be larger than the
critical chain length \(\mathrm{I}_{\mathrm{c}}\). We thus obtain for
the critical packing parameter \(P\)

\[
P=\frac{v}{a_{0} l_{c}}<\frac{1}{3}
\]

\begin{enumerate}
\def\labelenumi{(\Alph{enumi})}
\setcounter{enumi}{1}
\tightlist
\item
  Cylindrical micelles:
  \pandocbounded{\includegraphics[keepaspectratio]{soft-matter_files/mediabag/2025_01_27_a9af280ad1234567891011121314151617181920212223242526272829303132333435363738394041424344.jpg}}
\end{enumerate}

For a cylindrical micelle the total volume and surface area are given by
\(N v=\pi R^{2} L\) \(N a_{0}=2 \pi R L\)
\(\therefore \frac{v}{a_{0}}=\frac{R}{2}<\frac{l_{c}}{2}\) again using
\(R<I_{C}\).

We thus obtain for the critical packing parameter \(P\)
\(\frac{1}{3}<\frac{v}{a_{0} l_{c}}<\frac{1}{2}\)

Below the lower limit spherical micelles are formed.

\subsubsection{Bilayers:}\label{bilayers}

\pandocbounded{\includegraphics[keepaspectratio]{soft-matter_files/mediabag/2025_01_27_a9af280ad123456789101112131415161718192021222324252627282930313233343536373839404142434445.jpg}}

For a bilayer the total volume and surface area are given by

\[
\begin{aligned}
& N v=A D \\
& N a_{0}=2 A \\
& \frac{v}{a_{0}}=\frac{D}{2}<l_{c} \quad\left(\text { using } D<2 l_{c}\right)
\end{aligned}
\]

We thus obtain for the critical packing parameter \(P\)
\(\frac{1}{2}<\frac{v}{a_{0} l_{c}}<1\) Below the lower limit
cylindrical micelles are formed.

\subsection{GLASSES}\label{glasses}

Many fluids, whether simple liquids like argon or complex liquids such
as colloids and polymers, can be quenched to temperatures well below
their equilibrium freezing point without crystallization occurring on
experimental timescales. In thermodynamic terms this is interpreted as a
failure of the system to reach its true equilibrium (minimum free
energy) state, namely the ordered crystalline phase. Such
out-ofequilibrium systems are known as glasses. In terms of their
structure (eg as measured by a radial distribution function, \(g(r)\),
they are practically indistinguishable from a liquids, i.e.~they exhibit
some short ranged order, but no long ranged order. However, their
dynamic behaviour is quite different from that of liquids; glasses
exhibit very slow relaxation because molecules cannot easily diffuse. As
a result, glasses do not flow (on experimental timescales) and therefore
their mechanical properties, i.e.~their response to stress and shearing,
is more akin to that of solids.

\subsubsection{Glass forming systems}\label{glass-forming-systems}

Glasses may be formed if the cooling rate is so fast that the liquid
does not have time to crystallize , or alternatively because the
molecules have some feature to their structure or bonding that inhibits
the formation of a crystalline phase. Many glass forming materials are
found in nature. Below we describe just a few.

Elements: Sulphur, Selenium and Phosporous readily form glasses.
Metallic alloys: liquid metals can form glasses if quenched very rapidly
( \(10^{6} \mathrm{~K} \mathrm{~s}^{-1}\) ). Such glasses find
applications in recording heads and electrical transformers. Polymer
glasses: Due to entanglement effects, polymers form glasses very easily.
Examples from everyday life are polycarbonates (eyeglasses, shatterproof
windows) and polymethacrylate (plastic pipes and tubes). Oxide glasses:
Familiar examples are \(\mathrm{SiO}_{2}, \mathrm{Na}_{2} \mathrm{O}\)
and CaO (all components of window glass). Organic glasses: Sucrose
solution forms a glass used in boiled sweets. Toluene and methanol also
readily form glasses.

\subsubsection{Relaxation time and
viscosity}\label{relaxation-time-and-viscosity}

Let us first consider relaxation in liquids. We have already seen that
at high volume fractions a particle is effectively trapped by its
nearest neighbours, within a `cage'. To escape the cage, a particle must
jump. Let us now consider the characteristic time \(\tau\) between such
jumps and its dependence on temperature. Empirically, one finds that

\[
\tau^{-1} \sim v \exp \left(-\frac{\varepsilon}{k_{B} T}\right)
\]

Here \(v\) is the typical vibration frequency for a particle rattling
around in its cage. This can also be regarded as the frequency with
which the particle attempts to escape its cage. The probability of an
escape attempt succeeding depends on a Boltzmann factor, i.e.~the
exponential of the ratio of the energy cost associated with climbing the
cage ``walls'', \(\varepsilon\), to the typical thermal energy
\(k_{B} T\).

It can be shown (cf.~PH10002), that the relaxation time \(\tau\) is
proportional to another quantity, the viscosity \(\eta\), which is a
measure of the response of a material to a shear stress, or more loosely
speaking a measure of the ability of a liquid to flow. Thus
\(\tau \propto \eta\) and it follows that

\[
\eta=\frac{G_{0}}{v} \exp \left(\frac{\varepsilon}{k_{B} T}\right)
\]

where \(G_{0}\) is a proportionality constant. This characteristic
exponential temperature dependence is known as Arrhenius behaviour.
Loosely speaking the viscosity provides a measure of the ability of a
liquid to flow.

While Arrhenius behaviour is observed for the viscosity of liquids at
high temperature, things are very different at low temperature. The
experimentally measured viscosity and the associated configurational
relaxation time exhibits a temperature dependence that strongly departs
from that implied by the characteristic frequency of vibrations \(v\)

\[
\eta=\eta_{0} \exp \left(\frac{B}{T-T_{0}}\right)
\]

The empirical finding, known as the Vogel-Fulcher law implies that the
viscosity diverges exponentially at the finite temperature \(T_{0}\). In
practical terms, this means that as \(T_{0}\) is approached from above,
one reaches some temperature \(T_{g}\) (the glass transition
temperature) at which the time for a configuration to relax becomes
comparable to the timescale of the experiment itself \(\tau_{\exp }\).
At this point, the system falls out of equilibrium with respect to
rearrangements of the particle configurations.

\subsubsection{Characterising the glass
transition}\label{characterising-the-glass-transition}

The glass transition temperature \(T_{g}\) is that temperature at which
the system falls out of equilibrium on the experimental timescale
\(\tau_{\text {exp. }}\). Clearly therefore, if one does a long
experiment, in which the temperature is lowered slowly, the system will
have a greater time to relax at each successive temperature and will
stay in equilibrium to a lower temperature. Thus the experimental glass
transition temperature depends on the rate at which we do the
experiment. To illustrate this consider a measurement of some structural
quantity (such as the sample volume) as a function of temperature.
Various scenarios for the behaviour of the volume as the temperature are
shown in the following figure
\pandocbounded{\includegraphics[keepaspectratio]{soft-matter_files/mediabag/2025_01_27_a9af280ad12345678910111213141516171819202122232425262728293031323334353637383940414243444546.jpg}}

At the glass transition temperature, the curve exhibits a well defined
discontinuity in it slope, rather like what happens at an equilibrium
phase transition such as freezing or boiling. The reason why the glass
transition is not a true thermodynamic phase transition is that the
temperature at which the discontinuity occurs depends on the history of
the sample, i.e.~how rapidly it is cooled. We say that the glass
transition is a kinetic phase transition brought on by the divergence of
the structural relaxation time.

\subsubsection{A simple picture of the glass
transition}\label{a-simple-picture-of-the-glass-transition}

Theories of the glass transition are a matter of intense current
research and it is probably fair to say that the detailed physics is
still not well understood in a comprehensive fashion. Nevertheless
simple models offer some insight, in qualitative if not quantitative
agreement with experiment. One such theory is based on the idea of
cooperativity. In a liquid at high temperatures, a molecule can diffuse
simply by moving to occupy the space made available by the random local
motions of its neighbours. At low temperatures and high volume
fractions, the local motion of neighbours is insufficent to allow
diffusion. Instead a number of neighbours must move cooperatively in
order to make space for a molecule to move. The minimum number of
molecules that have to move in unison in order for diffusion of a
molecule to take place gives rise to the concept of a cooperative
rearranging region. As the temperature is lowered the size of this
region increases, diverging (with the viscosity) at the Vogel-Fulcher
temperature \(T_{0}\)
\pandocbounded{\includegraphics[keepaspectratio]{soft-matter_files/mediabag/2025_01_27_a9af280ad1234567891011121314151617181920212223242526272829303132333435363738394041424344454647.jpg}}
\pandocbounded{\includegraphics[keepaspectratio]{soft-matter_files/mediabag/2025_01_27_a9af280ad123456789101112131415161718192021222324252627282930313233343536373839404142434445464748.jpg}}

If we suppose that the energy barrier for a single molecule to move is
\(\mu\) and there are \(z\) molecules in the cooperatively rearranging
region, then the thermal activation barrier for motion is

\[
\tau^{-1} \sim v \exp \left(-\frac{z \mu}{k_{B} T}\right)
\]

with \(v\) the microscopic vibration frequency. Within this model, the
non Arrhenius behaviour of the relaxation time (see above) derives from
the increase in the number of molecules that have to move cooperatively
as temperature decreases, i.e.~the \(T\) dependence of \(z\).

\subsection{LITERATURE}\label{literature}

The best overall text is: R.A.L Jones, Soft Condensed Matter, Oxford
University Press. Shelfmark {[}539.2 Jon{]}.

Additionally, the following more specialised texts should also be
useful. They can be found in the University Library under the stated
shelfmark.

\subsubsection{Colloids}\label{colloids-1}

\begin{enumerate}
\def\labelenumi{(\arabic{enumi})}
\tightlist
\item
  D.F.Evans, H.Wennerström: The Colloidal Domain - Where Physics,
  Chemistry, Biology, and Technology Meet. VCH Publishers (1994).
  {[}541.18 Eva{]}
\item
  R.J.Hunter: Introduction to Modern Colloid Science. Oxford University
  Press (1993). 541.18 Hun{]}
\item
  W.B.Russel, D.A.Saville, W.R.Schowalter: Colloidal Dispersions
  Cambridge University Press (1989).{[}541.18 Rus{]}
\item
  D.H.Everett: Basic Principles of Colloid Science.
\end{enumerate}

Royal Society of Chemistry Paperbacks (1988) {[} 541.18 Eve{]}

\subsubsection{Polymers and surfactants}\label{polymers-and-surfactants}

\begin{enumerate}
\def\labelenumi{(\arabic{enumi})}
\tightlist
\item
  R.J. Young and P.A. Lovell: Introduction to polymers {[}678 You{]}.
\item
  M. Doi: Introduction to polymer physics {[}541.64 Doi{]}
\item
  J.Israelachvili, Intermolecular and Surface Forces, Academic Press
  (1992), Chs. 16 and 17
\end{enumerate}

\subsubsection{Glasses}\label{glasses-1}

\begin{enumerate}
\def\labelenumi{(\arabic{enumi})}
\tightlist
\item
  J. Zarzycki; Glasses and the vitreous state. Cambridge University
  Press (1991).
\end{enumerate}

\subsection{PROBLEMS}\label{problems}

\subsubsection{Relaxation time in atomic
fluids}\label{relaxation-time-in-atomic-fluids}

Spherical colloidal particles of radius 30 nm and density
\(1.35 \mathrm{~g} \mathrm{~cm}^{-3}\) are suspended in water
(temperature \(25^{\circ} \mathrm{C}\), density
\(1.0 \mathrm{~g} \mathrm{~cm}^{-3}\), viscosity \(1.0 \times 10^{-3}\)
Pa s ). Calculate the average distance from the origin along a given
axis travelled by a particle in 1 minute due to Brownian motion and
sedimentation, respectively. What are the displacements after 1 hour?

\subsubsection{Equipartition}\label{equipartition}

The equipartitioning theorem states that in thermal equilibrium each
generalised co-ordinate which occurs in the total system energy only as
a quadratic term contributes
\((1 / 2) \mathrm{k}_{\mathrm{B}} \mathrm{T}\) to the mean energy of the
system. Use the theorem to estimate the typical time it rakes atoms (or
small molecules) in an atomic liquid to move distances of the order of
their own size, i.e.~estimate the relaxation time.

\subsubsection{Gravitational length}\label{gravitational-length}

\begin{enumerate}
\def\labelenumi{(\alph{enumi})}
\tightlist
\item
  Find the typical length scale (`gravitational length') of the
  distribution of polystyrene particles (density
  \(\rho_{\mathrm{p}}=1.05 \times 10^{3} \mathrm{~kg} \mathrm{~m}^{-3}\)
  ) of radius 50 nm and \(1 \mu \mathrm{~m}\) respectively, suspended in
  water (density
  \(\rho_{\mathrm{w}}=1.00 \times 10^{3} \mathrm{~kg} \mathrm{~m}^{-3}\),
  viscosity \(1.0 \times 10^{-3} \mathrm{~Pa} \mathrm{~s}\) ) at room
  temperature. Hint: The gravitational length is the height at which the
  concentration falls to \(1 / \mathrm{e}\) of its greatest value.
\item
  If a container of height \(\mathrm{H}=4 \mathrm{~cm}\) contains a
  dilute suspension of these particles at an average concentration of
  \(n_{a}\) particles per unit volume, what are the actual
  concentrations (in terms of \(n_{a}\) ) at the top and bottom of the
  sample when sedimentation equilibrium has been reached.
\item
  If the above sample is shaken up to give an initially uniform
  concentration and then left undisturbed, estimate how long it will
  take the concentration profile to reach its equilibrium state (A
  surprisingly long time!)
\end{enumerate}

\subsubsection{Peclet number}\label{peclet-number}

A definition of a colloid is a particle small enough that its Brownian
motion is not dominated by gravity. Thus if the particle is raised a
distance equal to its radius, the increase in its gravitational
potential energy must be less than the thermal energy
\((3 / 2) \mathrm{k}_{\mathrm{B}} T\). Another measure of the relative
importance of the effects of Brownian motion and gravity, is the Peclet
number \(\mathrm{Pe}=\tau_{R} / \tau_{\text {sed }}\). Here
\(\tau_{\mathrm{R}}\) is the time taken by a particle to diffuse a
distance equal to its radius, and \(\tau_{\text {sed }}\) is the time
taken by it to sediment the same distance. When \(\mathrm{Pe} \ll 1\),
Brownian motion dominates; when \(\mathrm{Pe} \gg 1\), gravity
dominates. Show that the condition \(\mathrm{Pe}<1\) is more or less
equivalent to the definition of a colloid given above.

\subsubsection{Diffusion equation}\label{diffusion-equation}

Prove by direct substitution that
\(n(x, t)=C(4 \pi D t)^{-1 / 2} \exp \left(-x^{2} / 4 D t\right)\) is
indeed a solution to the 1 D diffusion equation
\(\partial n / \partial t=D^{\partial^{2} n} / \partial x^{2}\). Discuss
the meaning of the constant C .

\subsubsection{Mean square displacement}\label{mean-square-displacement}

Starting from the diffusion equation, deduce the mean square
displacement, \(\left\langle x^{2}\right\rangle=2 D t\), without using
the full solution to the diffusion equation. Note that \(n(x, t)\) is
the number density, from which the probability density of finding a
particle in the region \([\mathrm{x}, \mathrm{x}+\mathrm{dx}]\) during
the time interval \([\mathrm{t}, \mathrm{t}+\mathrm{dt}]\), can be
obtained by normalising with \(\int n(x, t) d x=N\) (the total number of
particles). Having this in mind, the mean square displacement is
\(\left\langle x^{2}(t)\right\rangle=\frac{\int_{-\infty}^{\infty} x^{2} n(x, t) d x}{\int_{-\infty}^{\infty} n(x, t) d x}=\frac{1}{N} \int_{-\infty}^{\infty} x^{2} n(x, t) d x\)
Hint: Multiply both sides of the 1D diffusion equation by \(x^{2}\) and
integrate by parts over \(x\) to get an o.d.e for
\(\left\langle x^{2}\right\rangle\) with t as the independent variable.

\subsubsection{Relaxation time in colloidal
suspensions}\label{relaxation-time-in-colloidal-suspensions}

Use \(\left\langle\underline{r}^{2}\right\rangle=6 D t\) and the
Stokes-Einstein relation to show that the time \(\tau_{\mathrm{R}}\) for
a particle to diffuse its own radius \(R\), scales as \(R^{3}\). The
relaxation time \(\tau_{R}\) is a good estimate for the time taken to
return to equilibrium after a disturbance. Estimate
\(\tau_{\mathrm{R}}\) for a suspension of spheres of radius
\(\mathrm{R} \approx 1 \mu \mathrm{~m}\), i.e.~at the upper limit of the
colloidal size range.

\subsubsection{Colloidal crystals}\label{colloidal-crystals}

Colloidal crystals are observed for volume fractions \(\phi\) between
0.545 and 0.74 . Show that the upper limit of \(\phi=0.74\) corresponds
to the closest packing in a fcc crystal. What is the average distance
between the colloidal particles for the lower limit \(\phi=0.545\) ?
Hint: to get the average consider a simple cubic lattice of the same
volume fraction

\subsubsection{Lagrange's expression for the radius of
gyration}\label{lagranges-expression-for-the-radius-of-gyration}

For N equal point masses at positions
\(\left\{\underline{R}_{j}\right\}\), the radius of gyration is defined
by
\(R_{g}^{2}=\frac{1}{N} \sum_{j=1}^{N}\left(\underline{R}_{j}-\underline{R}_{G}\right)^{2}\)
where \(\underline{R}_{G}=\frac{1}{N} \sum_{j=1}^{N} \underline{R}_{j}\)
is the position of the centre of mass. Show that

\[
R_{g}^{2}=\frac{1}{2 N^{2}} \sum_{j=1}^{N} \sum_{k=1}^{N}\left(\underline{R}_{j}-\underline{R}_{k}\right)^{2}
\]

Hint: Note that \(\frac{1}{N} \sum_{j=1}^{N} 1=1\)

\subsubsection{Mean length}\label{mean-length}

Calculate the mean end-to end length of a freely jointed polymer
consisting of \(10^{5}\) monomers of length \(4.322 \AA\). How does the
end-to-end length change if the valence angle of the chain is fixed at
\(108^{\circ}\) ?

\subsubsection{Real data}\label{real-data}

Light scattering measurements on dilute solutions of polystyrene, of
different molar masses M , at the theta temperature give the following
results for the mean square radius of gyration
\(\left\langle R_{g}^{2}\right\rangle\) :

\begin{longtable}[]{@{}
  >{\centering\arraybackslash}p{(\linewidth - 12\tabcolsep) * \real{0.1429}}
  >{\centering\arraybackslash}p{(\linewidth - 12\tabcolsep) * \real{0.1429}}
  >{\centering\arraybackslash}p{(\linewidth - 12\tabcolsep) * \real{0.1429}}
  >{\centering\arraybackslash}p{(\linewidth - 12\tabcolsep) * \real{0.1429}}
  >{\centering\arraybackslash}p{(\linewidth - 12\tabcolsep) * \real{0.1429}}
  >{\centering\arraybackslash}p{(\linewidth - 12\tabcolsep) * \real{0.1429}}
  >{\centering\arraybackslash}p{(\linewidth - 12\tabcolsep) * \real{0.1429}}@{}}
\toprule\noalign{}
\begin{minipage}[b]{\linewidth}\centering
\(\mathrm{M}\left(10^{6} \mathrm{~g} \mathrm{~mol}^{-1}\right)\)
\end{minipage} & \begin{minipage}[b]{\linewidth}\centering
4.04
\end{minipage} & \begin{minipage}[b]{\linewidth}\centering
1.56
\end{minipage} & \begin{minipage}[b]{\linewidth}\centering
1.20
\end{minipage} & \begin{minipage}[b]{\linewidth}\centering
1.06
\end{minipage} & \begin{minipage}[b]{\linewidth}\centering
0.626
\end{minipage} & \begin{minipage}[b]{\linewidth}\centering
0.394
\end{minipage} \\
\midrule\noalign{}
\endhead
\bottomrule\noalign{}
\endlastfoot
\(<R_{g}^{2}>\left(\mathrm{nm}^{2}\right)\) & 3260 & 1210 & 928 & 770 &
484 & 305 \\
\end{longtable}

\begin{enumerate}
\def\labelenumi{(\alph{enumi})}
\tightlist
\item
  Show that these results are consistent with
  \(\left\langle R_{g}^{2}\right\rangle^{1 / 2}=B M^{1 / 2} \quad\left(M=N M_{M}\right.\)
  where \(N\) is the number of monomers in the chain and \(M_{M}\) is
  the molar mass of one monomer;
  \(M_{M}=100 \mathrm{~g} \mathrm{~mol}^{-1}\) ) and estimate the value
  of \(B\).
\item
  Given that the `size' of a styrene monomer is about
  \(\mathrm{b}_{0}=0.23 \mathrm{~nm}\), calculate the factor \(C\) which
  represents the effect of restricted bond angles in
  \(\left\langle R^{2}\right\rangle=C N b_{0}^{2}\). From \(C\),
  calculate the bond angle \(\theta\) which would apply if polystyrene
  can be represented by a freely-rotating chain. {[}Remember
  \(\left.\left\langle R_{g}^{2}\right\rangle=\frac{1}{6}<R^{2}\right\rangle\)
  {]}
\item
  Calculate \(\left.<R_{g}^{2}\right\rangle^{1 / 2}\) for a polystyrene
  chain of molar mass \(1 \times 10^{7} \mathrm{~g} \mathrm{~mol}^{-1}\)
  at the theta temperature.
\item
  For the chain of (c), estimate the fraction of the volume
  \(\left.(4 \pi / 3)<R_{g}^{2}\right\rangle^{3 / 2}\) which is actually
  occupied by styrene monomers (assume that the density of styrene when
  polymerised is about \(1 \mathrm{~g} \mathrm{~cm}^{-3}\) ). What is
  the overlap concentration \(c^{*}\) (in mass per volume)? How do you
  expect \(c^{*}\) to change upon an increase (decrease) in temperature?
\end{enumerate}

\subsubsection{Micelles}\label{micelles}

The volume of a linear hydrocarbon chain with \(n\) carbon atoms is
given by \(v=(27.4+26.9 n) \times 10^{-3} \mathrm{~nm}^{3}\), and its
critical chain length is \(l_{c}=(0.154+0.1265 n)\) \(n m\). An
amphiphile has an anionic head group with an optimum head group area in
aqueous solution of \(a_{0}=0.65 \mathrm{~nm}^{2}\). (i) What shape
micelles are formed by amphiphiles with linear hydrocarbon chains having
\(\mathrm{n}=10\) ? (ii) What is the average size and aggregation number
of each micelle?

\subsubsection{Viscosity}\label{viscosity}

For polystyrene, the variation of viscosity with temperature follows the
Vogel-Fulcher law \(B=710\) and \(T_{0}=50^{\circ} \mathrm{C}\). Plot
the function \(\eta / \eta_{0}\) in the temperature range
\(80-150^{\circ} \mathrm{C}\). By what factor does the viscosity and
relaxation time vary between the temperatures of
\(100-140^{\circ} \mathrm{C}\) ?

\subsubsection{The glass transition}\label{the-glass-transition}

For polystyrene, a relaxation time associated with configurational
rearrangements, \(\tau_{\text {config, }}\), follows a Vogel-Fulcher
law,

\[
\tau_{\text {confg }}=\tau_{0} \exp \left(\frac{B}{T-T_{0}}\right)
\]

Where \(\tau_{0}, \mathrm{~B}=710\) and \(T_{0}=50^{\circ} \mathrm{C}\)
are constants. A value of the experimental glass transition temperature
is measured with an experiment carried out at an effective timescale
\(\tau_{\text {exp }}=1000\) s and found to be
\(101.4^{\circ} \mathrm{C}\). (a) Another experiment is carried out at
an effective timescale of \(10^{5} \mathrm{~s}\). What is the value of
the glass transition temperature obtained from this experiment? (b) On
what timescale must an experiment be carried out if it is to measure a
glass transition temperature within \(10^{\circ} \mathrm{C}\) of the
Vogel-Fulcher temperature \(T_{0}\). Is this practically possible?




\end{document}
